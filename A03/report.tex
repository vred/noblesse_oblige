\documentclass[10pt]{article}
\usepackage[margin=1in]{geometry}
\usepackage{fancyhdr}
\usepackage{calc}
\pagestyle{fancy}
\fancyhf{}
\setcounter{secnumdepth}{-1}

\begin{document}

\title{198:416 A03}
\author{Eric Cuiffo, Nick Palumbo, Val A. Red, Aiser Sheikh}

\fancyhead{}
\fancyhead[LO]{\bfseries Eric Cuiffo, Nick Palumbo, Val A. Red, Aiser Sheikh}
\fancyhead[RO]{198:416 A03}
.\\
Eric Cuiffo, Nick Palumbo, Val A. Red, Aiser Sheikh\\
6 May 2013 \\
198:416 Operating Systems, A03. \\
\textbf{\textit{Simple Network File System}}

\section{Introduction}

Our group, consisting of Eric Cuiffo, Nick Palumbo, Val A. Red, and Aiser Sheikh, built a 
the client and server applications for a \textbf{\textit{Simple Network File System (SNFS) }}. We 
accomplished this utilizing separate executables, \texttt{serverSNFS} and \texttt{clientSNFS}. 
The former, server-side executable is a multithreaded program that handles file 
management and requests; the latter, client-side executable communicates between 
user-end applications and the server over IP/TCP protocols. This utilizes FUSE to 
interpret and relay file system operations over network, then receiving the 
server's response to client-side applications.  \\

%Work we've done, difficulties we faced, and use cases for our file system. 
% 	Indicate group member responsibilities.

\section{I. Our Work on the SNFS Implementation :}

For communication between the client and server-side applications, we decided a major 
point of our implementation would be to send data over raw sockets by converting all 
messages to the \texttt{char*} format. This appropriately fit the descriptor of 
\textit{simple} network file system  by being the most elegant method to 
transfer data and requests over raw sockets.  As such, much of our code consists of 
simple character array and value manipulations corresponding to the data transferred 
over sockets. \\

\texttt{serverSNFS.c} \\

Our implementation of serverSNFS polls as a \texttt{while(1)} loop, 
polling for any requests sent over a designated port number to pass to the \textbf{request handler.} 
The request handler itself is a series of if-else statements that parse the the first 
argument to interpret what file system function is being requested.  \\

Each argument required by the server to implement each function is delimited by commas, 
our implementation then tokenizes this to truncate the arguments. \\

Multithreading is handled such that each time a request is sent to the server it 
dynamically spawns a new thread calling our \texttt{request\_handler} function, which 
then receives the data and switches on the name of the system call. As such, \textbf{the first 
argument sent over the socket is \textit{always} the system call.} \\

\texttt{clientSNFS.c} \\

FUSE, \textbf{F}ilesystem in \textbf{U}ser\textbf{S}pac\textbf{E}, was a large part of our 
clientSNFS implementation, as it enables us to efficiently and securely implement a client 
application in the userspace that can remotely mount the filesystem on serverSNFS. Much of our 
code mediates between default FUSE operations and our serverSNFS implementation, enabling 
anyone in the user-end running clientSNFS to use file operations remotely. \\

As such, clientSNFS features analagous file operations to those on serverSNFS via FUSE. Consistent 
with our implementation for serverSNFS, it utilizes the same \texttt{char*} implementation for 
core functions, also extensively utilizing character-string manipulation functions.    

\newpage

\section{II. Difficulties Faced}

Most of the difficulties we faced were regarding memory management, as at several of points we would 
reach segmentation faults. As such, we were certain to build our program incrementally and avoid 
losing track of which part of our code was responsible for said segmentation faults. We were especially 
careful after implementing multithreading to check for any possible issues with our code. Particularly, 
\texttt{readdir} and \texttt{opendir} were tricky to implement with multithreading, as they would begin 
to cause segmentation faults while they worked fine when single-threaded. Nevertheless, after correcting 
issues with the order of our \texttt{char*} and \texttt{str} operations, we were able to debug the issues 
and fully implement those functions without error on execution. \\ 

One of the challenges we faced was attempting to use \texttt{memcpy} to change structs into \texttt{char*} 
and send them over to the client and vice versa. As the data was sent over, only empty structs resulted. 
We eventually decided on using strcpy and sending the data as strings because it was the 
easiest way although slightly less efficient. Nevertheless, the tradeoff of better management of 
data and bypassing this issue we encountered made it an elegant solution, again stressing the 
value of our Simple Network File System. 

\section {III. Use cases}

To initialize the serverSNFS application, one can enter the following on a terminal in the 
\texttt{serverSNFS} directory: 
\begin{verbatim}
	./serverSNFS -mount FSDirectory/ -port 6590
\end{verbatim}

If on the same computer, one can also run the clientSNFS application and continue to 
make directories, create and remove files, as in the following string of commands: 
\begin{verbatim}
	./clientSNFS -mount NetDirectory/ -port 6590 -address 127.0.0.1 
	cd NetDirectory/ 
	echo "Sample file one" > file1.txt
	mkdir newdirectory 
	cd newdirectory
	echo "another file" > file.txt 
	rm file.txt 
	cd ..
\end{verbatim} 

This use case demonstrates several of possibilities with our file system. You can create, open, 
and read directories and files. Several of base unix functions are also available for the user 
to use with our file system, such as ``echo'', ``cat'', ``vi'', and many other applications. 
Users are even able to remove files as we've implemented \texttt{unlink}, just not directories. \\

Essentially, our SNFS supports most basic file system operations, such as the following:
\begin{verbatim}
	create,open,write,close,truncate,opendir,readdir,releasedir,mkdir,getattr,unlink
\end{verbatim}

\section {IV. Group Member Responsibilities}

We shared many of the responsibilities over the project, meeting over several of days to plan 
and code our implementation. Nick Palumbo was the lead programmer, providing the direction and 
skill in implementing most of the core functionality. Eric Cuiffo, Val Red, and Aiser Sheikh 
shared roles in helping develop some of the functions, debugging the code, and preparing the 
final implementation. In addition, Eric implemented error-checking. Val Red and AIser Sheikh also 
authored the report, to which Nick and Eric were contributors. 

\section {V. Conclusion}

In our implementation, we learned a lot about communication over network protocols, file 
systems, and memory management; we applied all of this in creating a basic file system that 
worked over network between a server and client similar to Samba (SMB) utilizing simple-but-effective 
\texttt{char*} implementations and manipulations with a FUSE-assisted client program that users 
can utilize to remotely view and operate files on the server. 

Overall, we successfully implemented our \textbf{\textit{Simple Network File System (SNFS) }} 
with core functionalities that enable basic file systems to be executed on a remotely mounted 
drive between a server and a client application over a network.

\end{document}
